\documentclass{article}
\usepackage{amsmath}
\usepackage{graphicx}
\usepackage{geometry}
\usepackage{parskip}
\usepackage{float}  % Allows the use of [H] for exact figure placement

\geometry{margin=1in}

\title{Augmented Reality with Aruco Markers}
\author{Jonas Alber, Schweitzer Tim}
\date{November 9, 2024}

\begin{document}

\maketitle

\section*{Task 1.1: Algorithm Development and Evaluation}

\subsection*{Goal}
The objective is to detect Aruco markers in a dataset of images and overlay a poster image on all detected markers. The poster must be placed in a consistent position across all images with a size comparable to the provided images.

\subsection*{Methodology}
\begin{enumerate}
    \item Aruco markers were detected using the \texttt{cv2.aruco.detectMarkers} function with the \texttt{DICT\_6X6\_250} dictionary.
    \item A perspective transformation matrix aligned the poster image with the marker positions.
\end{enumerate}

\begin{figure}[H]  % Changed to [H] to force figure placement here
    \centering
    \includegraphics[width=0.3\textwidth]{transformed_poster.png}  % Adjusted size to 30% width
    \caption{Transformed poster overlaid onto the marker position.}
\end{figure}

\begin{enumerate}
    \item A binary mask was created to isolate the overlay region.
\end{enumerate}

\begin{figure}[H]  % Changed to [H]
    \centering
    \includegraphics[width=0.3\textwidth]{mask.png}  % Adjusted size
    \caption{Mask created to isolate the region for poster overlay.}
\end{figure}

\begin{enumerate}[resume]
    \item The resized and transformed poster is added to the cleared region using a bitwise OR operation (\texttt{cv2.bitwise\_or}). An optional border (\texttt{cv2.polylines}) can be drawn around the overlay area for emphasis and can be extended for evaluation.
\end{enumerate}

\begin{figure}[H]  % Changed to [H]
    \centering
    \includegraphics[width=0.3\textwidth]{final_image.png}  % Adjusted size
    \caption{Final image with the poster overlaid and perspective lines extended for evaluation.}
\end{figure}

\subsection*{Comparisons of Original and Augmented Frames}
To demonstrate the algorithm’s performance, the following comparison shows the augmented pictures with extended perspective lines for evaluation and comparison. The images below highlight the accuracy and consistency of the poster placement, as well as the effectiveness of the perspective transformation.

\subsection*{Task 1.2: Using Your Own Recordings}

\subsection*{Goal}
To validate the algorithm further, new scenes were recorded with Aruco markers, emphasizing strong perspective transformations. Augmented results are compared to the original scenes.

\subsection*{Challenges and Improvements}

\subsubsection*{Challenges}
\begin{itemize}
    \item Scaling and lighting issues in the dataset and recorded images impacted marker detection.
    \item Some frames exhibited distortion, affecting poster alignment.
\end{itemize}

\subsubsection*{Potential Improvements}
\begin{itemize}
    \item Utilize multiple markers for enhanced stability and accuracy.
    \item Implement camera calibration for distortion-free marker detection.
    \item Images where the marker quality is higher improve detection reliability.
\end{itemize}

\subsection*{Conclusion}
The algorithm achieved consistent poster placement across all dataset frames and user-recorded scenes.

\end{document}
